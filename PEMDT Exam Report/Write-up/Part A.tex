\section{Power Electronics}
    \subsection{Fundamental Building Blocks and Design}
        The fundamental building block in power electronic converters is the power semiconductor device. These devices are usually diodes or transistors, enabling precise control over electrical current and voltage flow. Given that they are typically used for switching operations, they allow for converting electrical power from one form to another, such as from AC to DC or, in our case, from one DC voltage level to another. 

        As the system is required to decrease the DC voltage level, two main voltage regulators can be used: a linear regulator or a switched-mode power supply (SMPS). The appropriate topology for compact DC-to-DC converters with high efficiencies, commonly known as buck converters, typically uses field effect transistors like MOSFETs and IGBTs in switch mode. However, these converters require passive components to smooth and filter the outputs of the switch. Designing this type of power electronic converter requires a systematic approach which incorporates the following steps:

        \begin{enumerate}
            \item For a given switching frequency, a pulse width modulated (PWM) waveform is generated using a duty cycle (\(\delta\)) derived from the stated input and output voltage. 
            
            \item The appropriate power semiconductor device has to be selected for switching. This primarily depends on the voltage/current rating, switching speed and losses of the chosen transistor and how well it adheres to the given system. This switch is fed an input from the low-voltage DC bus and the PWM to create a modulated voltage waveform.

            \item A diode is set up anti-parallel to the input voltage. The appropriate diode must be selected to allow free-wheeling current flow in the correct direction when the switch is off and blocks the reverse voltage when the switch is on.

            \item The capacitor and inductor are passive components set up in parallel and series to the input voltage, respectively. Their capacitance and inductance values have to be derived to filter, smooth and shape voltage and current fluctuations, respectively.

            \item The correct heat sink has to be selected for thermal management that implements appropriate heat dissipation and cooling mechanisms to keep the switching device within its safe operating temperature range. 

            \item The final step requires the design to be optimised. Analysis of the system with different switching frequencies generally shows that a higher switching frequency offers advantages such as the reduced size of inductors and capacitors, leading to compact designs and higher power density. However, it has disadvantages, including increased switching losses, higher electromagnetic interference (EMI), and potentially larger heat sink requirements. The choice of switching frequency should consider trade-offs based on the application's specific requirements, balancing factors such as efficiency, size constraints, EMI concerns, and thermal management.
        \end{enumerate}

        \textit{Add a bit on the duty cycles chosen explaining how you can get \(\delta_{min} \& \delta_{max}\) from the values provided}

    \subsection{SMPS vs Conventional Voltage Regulators}
        SMPS is more efficient than conventional voltage regulators for several reasons. 
        Firstly, a conventional voltage regulator regulates the output voltage by controlling the current flow between the input and output terminals. This is achieved using a pass transistor (typically a bipolar junction transistor or a MOSFET) connected in series between the input and output terminals. The pass transistor acts as a variable resistance, fundamentally adjusting its impedance to drop the excess voltage across it and maintain a stable output voltage. The downside is that the series transistor is continually biased in its linear region, dissipating power in the form of heat. Since all the load current must pass through the series transistor, this results in poor efficiency, wasted \(iV\) or \(i^2R\) power and continuous heat generation around the transistor. In contrast, SMPS operate by rapidly switching the input voltage on and off. When the switching transistor is on and conducting current, the voltage drop across it is at its minimal value, and when the transistor is off, there is no current flow through it. So the transistor is acting like an ideal on/off switch. This means that the majority of the \(iV\) power losses occur during the switching phase, which is very short in comparison to the constant power loss in conventional voltage regulators.
        Secondly, something worth mentioning is the space efficiency of SMPS compared to conventional voltage regulators. Due to the large power dissipation of conventional voltage regulators, the heat sinks on them are usually much larger. As well as this, SMPS is able to use much smaller components due to the higher switching frequency and is, therefore, more energy dense and more efficient in terms of its overall size-to-power loss ratio.

    \subsection{MOSFET vs IGBT}
        Insulated Gate Bipolar Transistors (IGBT) and Metal Oxide Semiconductor Field Effect Transistors (MOSFET) are both semiconductor devices, but they have key differences, as shown in Table \ref{tab: SMPS Comparison}.

        \begin{table}[tbh!]
\caption{Fundamental differences between MOSFETs and IGBTs}
\label{tab: SMPS Comparison}
\resizebox{\columnwidth}{!}{%
\begin{tabular}{|p{2.2cm}|p{7cm}|p{8cm}|}
\hline
\textbf{} & \textbf{MOSFET} & \textbf{IGBT} \\ \hline
\textbf{Name} & Metal Oxide Semiconductor Field Effect Transistor & Insulated Gate Bipolar Transistor \\ \hline
\textbf{Structure} & MOSFETs consist of a gate electrode separated from the semiconductor channel by an insulating layer. When a voltage is applied to the gate, the channel controls the current flow between the source and drain terminals. & IGBTs consist of four layers: p+ collector, n- drift, n- buffer, and p+ emitter. The IGBT operates based on the voltage applied to the gate, controlling the current flow between the collector and emitter. The P-N-P-N structure allows the IGBT to combine the advantages of both MOSFETs and bipolar transistors. \\ \hline
\textbf{Conductivity} & MOSFETs are unipolar devices, meaning they primarily rely on the movement of either electrons (in n-channel MOSFETs) or holes (in p-channel MOSFETs) in the channel. The channel conductivity is modulated by the electric field generated by the voltage applied to the gate terminal. & IGBTs are bipolar devices, meaning they operate based on the movement of both electrons (in the n-type regions) and holes (in the p-type regions). The P-N-P-N structure allows for the flow of current through both electrons and holes. \\ \hline
\textbf{Voltage Rating} & MOSFETs are commonly used in low to medium-voltage applications. The voltage ratings of MOSFETs can range from a few volts to a few hundred volts. This is typical because they have lower voltage distribution and thinner depletion regions, given they have fewer layers. & IGBTs are designed for high-voltage applications. The voltage ratings of IGBTs range from a few hundred volts to several kilo-volts. IGBTs are suitable for applications that require high voltage blocking capability. \\ \hline
\textbf{Current Rating} & MOSFETs are generally used in low to medium current applications. The current ratings vary depending on the specific MOSFET model and design. & IGBTs have higher current handling capabilities compared to MOSFETs. They are capable of handling medium to high current levels. IGBTs are commonly used in high-power applications where significant current flow is required. \\ \hline
\textbf{Switching Speed} & MOSFETs offer faster switching speeds than IGBTs. Due to their unipolar nature, MOSFETs can quickly transition between on and off states, making them suitable for high-frequency applications. & IGBTs have slower switching speeds compared to MOSFETs. The presence of bipolar junction transistors causes a ``tail-off" effect when they turn off as the electrons and holes have to recombine; this results in longer turn-off times. \\ \hline
\textbf{Application Preferences} & MOSFETs are commonly used in low to medium-power applications. They find widespread use in consumer electronics, such as smartphones, laptops, and power supplies. They are also used in motor control, LED lighting, and other applications that require efficient power switching. & IGBTs are preferred in high-power applications. They are used in industrial drives, motor control systems, renewable energy systems, and high-power inverters. IGBTs excel in applications that require high voltage blocking, high current handling, and efficient power switching. They are well-suited for applications where power efficiency and reliability are key. \\ \hline
\end{tabular}%
}
\end{table}

    \subsection{Transistor Selection}

    \subsection{Diode Selection}