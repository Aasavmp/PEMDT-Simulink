\section{Power Electronics}
    \subsection{Fundamental Building Blocks and Design}
        The fundamental building block in power electronic converters is the power semiconductor device. These devices are usually diodes or transistors, enabling precise control over electrical current and voltage flow. Given that they are typically used for switching operations, they allow for converting electrical power from one form to another, such as from AC to DC or, in our case, from one DC voltage level to another. 

        As the system is required to decrease the DC voltage level, two main voltage regulators can be used: a linear regulator or a switched-mode power supply (SMPS). The appropriate topology for compact DC-to-DC converters with high efficiencies, commonly known as buck converters, typically uses field effect transistors like MOSFETs and IGBTs in switch mode. However, these converters require passive components to smooth and filter the outputs of the switch. Designing this type of power electronic converter requires a systematic approach which incorporates the following steps:

        \begin{enumerate}
            \item For a given switching frequency, a pulse width modulated (PWM) waveform is generated using a duty cycle (\(\delta\)) derived from the stated input and output voltage. 
            
            \item The appropriate power semiconductor device has to be selected for switching. This primarily depends on the voltage/current rating, switching speed and losses of the chosen transistor and how well it adheres to the given system. This switch is fed an input from the low-voltage DC bus and the PWM to create a modulated voltage waveform.

            \item A diode is set up anti-parallel to the input voltage. The appropriate diode must be selected to allow free-wheeling current flow in the correct direction when the switch is off and blocks the reverse voltage when the switch is on.

            \item The capacitor and inductor are passive components set up in parallel and series to the input voltage, respectively. Their capacitance and inductance values have to be derived to filter, smooth and shape voltage and current fluctuations, respectively.

            \item The correct heat sink has to be selected for thermal management that implements appropriate heat dissipation and cooling mechanisms to keep the switching device within its safe operating temperature range. 

            \item The max. and min. duty cycle (\(\delta\)) of the system should be calculated. This is done using the following equation:
            \begin{align}
                \delta = \frac{V_{out}}{V_{in}} && \delta_{max} = \frac{V_{inmax}}{V_{outmin}} = 0.545 && \delta_{min} = \frac{V_{inmin}}{V_{outmax}} = 0.321
            \end{align}

            \item The final step requires the design to be optimised. Analysis of the system with different switching frequencies generally shows that a higher switching frequency offers advantages such as the reduced size of inductors and capacitors, leading to compact designs and higher power density. However, it has disadvantages, including increased switching losses, higher electromagnetic interference (EMI), and potentially larger heat sink requirements. The choice of switching frequency should consider trade-offs based on the application's specific requirements, balancing factors such as efficiency, size constraints, EMI concerns, and thermal management.
        \end{enumerate}

    \subsection{SMPS vs Conventional Voltage Regulators}
        SMPS is more efficient than conventional voltage regulators for several reasons. 
        Firstly, a conventional voltage regulator regulates the output voltage by controlling the current flow between the input and output terminals. This is achieved using a pass transistor (typically a bipolar junction transistor or a MOSFET) connected in series between the input and output terminals. The pass transistor acts as a variable resistance, fundamentally adjusting its impedance to drop the excess voltage across it and maintain a stable output voltage. The downside is that the series transistor is continually biased in its linear region, dissipating power in the form of heat. Since all the load current must pass through the series transistor, this results in poor efficiency, wasted \(iV\) or \(i^2R\) power and continuous heat generation around the transistor. In contrast, SMPS operate by rapidly switching the input voltage on and off. When the switching transistor is on and conducting current, the voltage drop across it is at its minimal value, and when the transistor is off, there is no current flow through it. So the transistor acts like an ideal on/off switch. This means that most of the \(iV\) power losses occur during the switching phase, which is very short compared to the constant power loss in conventional voltage regulators.
        
        Secondly, the space efficiency of SMPS compared to conventional voltage regulators is worth mentioning. Due to the large power dissipation of conventional voltage regulators, the heat sinks on them are usually much larger. As well as this, SMPS can use much smaller components due to the higher switching frequency and is, therefore, more energy dense and more efficient in terms of its overall size-to-power loss ratio.

    \subsection{MOSFET vs IGBT}
        Insulated Gate Bipolar Transistors (IGBT) and Metal Oxide Semiconductor Field Effect Transistors (MOSFET) are both semiconductor devices, but they have key differences, as shown in Table \ref{tab: SMPS Comparison}.
        \begin{table}[tbh!]
\caption{Fundamental differences between MOSFETs and IGBTs}
\label{tab: SMPS Comparison}
\resizebox{\columnwidth}{!}{%
\begin{tabular}{|p{2.2cm}|p{7cm}|p{8cm}|}
\hline
\textbf{} & \textbf{MOSFET} & \textbf{IGBT} \\ \hline
\textbf{Name} & Metal Oxide Semiconductor Field Effect Transistor & Insulated Gate Bipolar Transistor \\ \hline
\textbf{Structure} & MOSFETs consist of a gate electrode separated from the semiconductor channel by an insulating layer. When a voltage is applied to the gate, the channel controls the current flow between the source and drain terminals. & IGBTs consist of four layers: p+ collector, n- drift, n- buffer, and p+ emitter. The IGBT operates based on the voltage applied to the gate, controlling the current flow between the collector and emitter. The P-N-P-N structure allows the IGBT to combine the advantages of both MOSFETs and bipolar transistors. \\ \hline
\textbf{Conductivity} & MOSFETs are unipolar devices, meaning they primarily rely on the movement of either electrons (in n-channel MOSFETs) or holes (in p-channel MOSFETs) in the channel. The channel conductivity is modulated by the electric field generated by the voltage applied to the gate terminal. & IGBTs are bipolar devices, meaning they operate based on the movement of both electrons (in the n-type regions) and holes (in the p-type regions). The P-N-P-N structure allows for the flow of current through both electrons and holes. \\ \hline
\textbf{Voltage Rating} & MOSFETs are commonly used in low to medium-voltage applications. The voltage ratings of MOSFETs can range from a few volts to a few hundred volts. This is typical because they have lower voltage distribution and thinner depletion regions, given they have fewer layers. & IGBTs are designed for high-voltage applications. The voltage ratings of IGBTs range from a few hundred volts to several kilo-volts. IGBTs are suitable for applications that require high voltage blocking capability. \\ \hline
\textbf{Current Rating} & MOSFETs are generally used in low to medium current applications. The current ratings vary depending on the specific MOSFET model and design. & IGBTs have higher current handling capabilities compared to MOSFETs. They are capable of handling medium to high current levels. IGBTs are commonly used in high-power applications where significant current flow is required. \\ \hline
\textbf{Switching Speed} & MOSFETs offer faster switching speeds than IGBTs. Due to their unipolar nature, MOSFETs can quickly transition between on and off states, making them suitable for high-frequency applications. & IGBTs have slower switching speeds compared to MOSFETs. The presence of bipolar junction transistors causes a ``tail-off" effect when they turn off as the electrons and holes have to recombine; this results in longer turn-off times. \\ \hline
\textbf{Application Preferences} & MOSFETs are commonly used in low to medium-power applications. They find widespread use in consumer electronics, such as smartphones, laptops, and power supplies. They are also used in motor control, LED lighting, and other applications that require efficient power switching. & IGBTs are preferred in high-power applications. They are used in industrial drives, motor control systems, renewable energy systems, and high-power inverters. IGBTs excel in applications that require high voltage blocking, high current handling, and efficient power switching. They are well-suited for applications where power efficiency and reliability are key. \\ \hline
\end{tabular}%
}
\end{table}

    \subsection{Transistor Selection}
        When designing this converter it was crucial to pick a transistor that adhered to the converters operational constraints and was as efficient as possible. Initially, all the available transistors were compared to the maximum continuous forward current (60A) and voltage (28V) of the system. It was found that only 4 transistors had a higher drain current or current carrying capacity than the system's requirements. These 4 transistors (2, 5, 8 \& 9) all had a higher breakdown voltage than the systems max. continuous voltage so would all operate in a region where they would not experience electrical breakdown and fail. They are highlighted in green on Table \ref{tab: transistor characterstics}.
        \begin{table}[tbh!]
\caption{Transistor Characteristics}
\label{tab: transistor characterstics}
\resizebox{\columnwidth}{!}{%
\begin{tabular}{|c|c|c|c|c|c|c|}
\hline
\textbf{} & \textbf{Type} & \textbf{\begin{tabular}[c]{@{}c@{}}Drain \\ Current {[}A{]}\\  @ 25\(^o\)C\end{tabular}} & \textbf{\begin{tabular}[c]{@{}c@{}}Breakdown \\ Voltage {[}V{]}\end{tabular}} & \textbf{\begin{tabular}[c]{@{}c@{}}Max. On-State\\  Resistance [m\(\Omega\)]\end{tabular}} & \textbf{\begin{tabular}[c]{@{}c@{}}Min. \\ Switching \\ Time {[}ns{]}\end{tabular}} & \textbf{\begin{tabular}[c]{@{}c@{}}Power \\ Loss {[}W{]} \\ @ 20kHz\end{tabular}} \\ \hline
\textbf{Transistor} & \multicolumn{1}{r|}{\textbf{Threshold}} & \cellcolor[HTML]{FCE4D6}\(\geq\)60 & \cellcolor[HTML]{FCE4D6}\(\geq\)28 &  &  & \\ \hline
1 & Power MOSFET & 15 & 500 &  &  &  \\ \hline
\rowcolor[HTML]{C6E0B4} 
2 & IGBT IXYA50N65C3 & 132 & 650 & 48 & 172 & 115.4 \\ \hline
3 & IGBT & 40 & 1200 &  &  &  \\ \hline
4 & MOSFET & 50 & 25 &  &  &  \\ \hline
\rowcolor[HTML]{C6E0B4} 
{\color[HTML]{FF0000} \textbf{5}} & {\color[HTML]{FF0000} \textbf{MOSFET IRFB7545PbF}} & {\color[HTML]{FF0000} \textbf{95}} & {\color[HTML]{FF0000} \textbf{60}} & {\color[HTML]{FF0000} \textbf{4.9}} & {\color[HTML]{FF0000} \textbf{171}} & {\color[HTML]{FF0000} \textbf{13.5}} \\ \hline
6 & MOSFET & 50 & 50 &  &  &  \\ \hline
7 & IGBT & 26.9 & 400 &  &  &  \\ \hline
\rowcolor[HTML]{C6E0B4} 
8 & MOSFET & 75 & 30 & 3.6 & 216 & 11.1 \\ \hline
\rowcolor[HTML]{C6E0B4} 
9 & MOSFET Silicon Carbide & 115 & 1200 & 16 & 313 & 44.7 \\ \hline
10 & IGBT & 40 & 600 &  &  &  \\ \hline
\end{tabular}%
}
\end{table}

        The 4 potential transistors were then compared to see if they could withstand the switching frequencies. Although transistor 2 \& 5 had the lowest switching time and therefore the highest potential switching frequency, all 4 transistors were capable of switching frequencies up to 3MHz so they could all be used based on this criteria.
        
        Although all 4 of these transistors can operate in a safe region the efficiency of the converter also has to be maximised. The efficiency of the system can be improved by minimising losses in the system, and in this case the transistor. The losses of the system can be worked out by adding the conduction loss (\(P_{conduction}\)) to the switching loss (\(P_{switch}\)) to give an overall power loss. The conduction and switching losses can be calculated using the following equations: 
        \begin{gather}
            P_{conduction} = I^2R_{DS_{on}} \delta_{max} \label{conduction loss} \\ P_{switch} = \frac{V f_{sw}}{2} [t_{on} + t_{off}] I, \label{switching loss}
        \end{gather}
        where \(I=60\)A, the system's forward current; \(R_{DS_{on}}\) is the on-state resistance; \(\delta_{max} = 0.545\) is the max. duty cycle; \(V\) is the drain-source voltage; \(f_{sw} = 20\)kHz is the switching frequency; and \(t_{on}/t_{off}\) is the time taken for the transistor to switch on and off respectively.
        
        Using Eq. (\ref{conduction loss}) \& (\ref{switching loss}) it is clear that transistor 5 \& 8 have the lowest losses, however, transistor 8 has a breakdown voltage that is very close to the systems operating voltage, so if the converter is designed with a safety factor of 1.5 to account for any fluctuations transistor 5 is the optimal choice (highlighted in red on Table \ref{tab: transistor characterstics}).

    \subsection{Diode Selection}
        As with the transistor, the diode was selected in a similar way. All the diodes were compared against the converters operational criteria (at the diode) of a 60A forward current and a 12V reverse voltage drop across the diode. As shown in Table \ref{tab: diode characterisitics}, only 4 diodes, highlighted in green, met the system's criteria. All 4 potential diodes had a similar average power loss. However, diode 6 (in red on Table \ref{tab: diode characterisitics}) has the lowest forward voltage, the minimum voltage across the diode needed to allow current to flow. Meaning it has the most effective and efficient operation during conduction. It also has the quickest reverse recovery time, the time it takes for a diode to transition from the conducting state to the non-conducting state when the polarity of the voltage across it is reversed. This meant that compared to the other potential diodes, it is able to withstand the highest switching frequency, which is ideal for a system using a MOSFET. However, this being said, all the potential diodes operate on the threshold of the system's forward current, and therefore, none of the diodes fully meet safety requirements if a safety factor of 1.5 times the max. current is used (rule of two-thirds).
        \begin{table}[tbh!]
\caption{Diode Characteristics}
\label{tab: diode characterisitics}
\resizebox{\columnwidth}{!}{%
\begin{tabular}{|c|c|c|c|c|c|c|}
\hline
\rowcolor[HTML]{FFFFFF} 
\textbf{} & \textbf{Name} & \textbf{\begin{tabular}[c]{@{}c@{}}Max. Forward \\ Current [A] \\ @ 125\(^o\)C \(\pm\) 15\(^o\)C\end{tabular}} & \textbf{\begin{tabular}[c]{@{}c@{}}Max. Reverse \\ Voltage [V]\end{tabular}} & \textbf{\begin{tabular}[c]{@{}c@{}}Forward Voltage [V] \\ @ Max. Current\end{tabular}} & \textbf{\begin{tabular}[c]{@{}c@{}}Reverse \\ Recovery \\ Time [ns]\end{tabular}} & \textbf{\begin{tabular}[c]{@{}c@{}}Average Power\\  Loss [W]\\  @ 60 A\end{tabular}} \\ \hline
\rowcolor[HTML]{FFFFFF} 
\textbf{Diode} & \multicolumn{1}{r|}{\cellcolor[HTML]{FFFFFF}\textbf{Threshold}} & \cellcolor[HTML]{FCE4D6}\(\geq\)60 & \cellcolor[HTML]{FCE4D6}\(\geq\)12 & - & - & - \\ \hline
\rowcolor[HTML]{FFFFFF} 
1 & VT4045BP-M3 & 40 & 45 & 0.51 & N/A & N/A \\ \hline
\rowcolor[HTML]{C6E0B4} 
2 & VS-E4PU6006L-N3 & 60 & 600 & 1.29 & 74 & 75 \\ \hline
\rowcolor[HTML]{FFFFFF} 
3 & \begin{tabular}[c]{@{}c@{}}VS-8TQ060-M3, \\ VS-8TQ080-M3, \\ VS-8TQ100-M3\end{tabular} & 8 & \begin{tabular}[c]{@{}c@{}}60, \\ 80, \\ 100\end{tabular} & 0.58 & N/A & 72 \\ \hline
\rowcolor[HTML]{FFFFFF} 
4 & SCS306AM & 6 & 650 & 1.35 & N/A & N/A \\ \hline
\rowcolor[HTML]{C6E0B4} 
5 & RFL60TZ6S & 60 & 650 & 1.3 & 60 & 75 \\ \hline
\rowcolor[HTML]{C6E0B4} 
{\color[HTML]{FE0000} \textbf{6}} & {\color[HTML]{FE0000} \textbf{STTH60R04}} & {\color[HTML]{FE0000} \textbf{60}} & {\color[HTML]{FE0000} \textbf{400}} & {\color[HTML]{FE0000} \textbf{0.95}} & {\color[HTML]{FE0000} \textbf{31}} & {\color[HTML]{FE0000} \textbf{75}} \\ \hline
\rowcolor[HTML]{FFFFFF} 
7 & E4D20120A & 54.5 & 1200 & 1.5 & 0 & N/A \\ \hline
\rowcolor[HTML]{FFFFFF} 
8 & UJ3D1250K2 & 50 & 1200 & 1.5 & 0 & N/A \\ \hline
\rowcolor[HTML]{FFFFFF} 
9 & C4D20120A & 26 & 1200 & 1.5 & 0 & N/A \\ \hline
\rowcolor[HTML]{C6E0B4} 
10 & DSEI60-06A(T) & 60 & 600 & 1.5 & 35 & N/A \\ \hline
\end{tabular}%
}
\end{table}

    \subsection{Heatsink Selection}
        The final part to select for the converter is the heatsink that will be used to control the chosen transistor's temperature. Heatsink 1 was found to be too big and could not be used effectively for the chosen transistor. Out of the remaining heatsinks, to determine which heatsink is best, the required thermal conductivity and resistance of the heatsink had to be calculated. This was done by first calculating the temperature at the heatsink using known values of the transistor's thermal resistivity and comparing this temperature to the ambient temperature, \(T_{ambient}\), as shown below:
        \begin{gather}
            T_{sink} = T_j - (R_{jc} + R_{cs})P_{loss} \\
            R_{\theta sink} = \frac{1}{C_{\theta sink}} = \frac{T_{sink} - T_{ambient}}{P_{loss}},
        \end{gather}
        where \(T_{sink}\) is the temperature at the heatsink; \(T_j\) is the max. junction temperature of the transistor; \(R_{jc}/R_{cs}\) are the transistor's thermal resistivity between the junction/case and case/heatsink respectively; \(P_{loss}\) is the power loss of the transistor; and \(R_{\theta sink}\) and \(C_{\theta sink}\) are the required thermal resistance and conductivity of the heatsink respectively.

        The required thermal resistivity of the heatsink at a switching frequency of \(f_{sw} = \) 20kHz and a duty cycle of \(\delta_{max} = 0.545\) was calculated to be \(R_{\theta sink} = 9.40^o\)C/W or \(C_{\theta sink} = 0.106 \)W/\(^o\)C.

        \begin{table}[tbh!]
\centering
\caption{Heatsink Characteristics}
\label{tab: heatsink charactersitics}
\resizebox{0.7\columnwidth}{!}{%
\begin{tabular}{|c|c|c|}
\hline
\textbf{Heatsink} & \textbf{Thermal Resistivity [\(^o\)C/W]} & \textbf{Thermal Conductivity [W/\(^o\)C]} \\ \hline
\multicolumn{1}{|r|}{\textbf{Threshold}} & \cellcolor[HTML]{FCE4D6}\(\leq\)9.4 & \cellcolor[HTML]{FCE4D6}\(\geq\)0.106 \\ \hline
2 & 10.6 & 0.094 \\ \hline
3 & 12.8 & 0.078 \\ \hline
\rowcolor[HTML]{C6E0B4} 
{\color[HTML]{FF0000} \textbf{4}} & {\color[HTML]{FF0000} \textbf{8.7}} & {\color[HTML]{FF0000} \textbf{0.115}} \\ \hline
\end{tabular}%
}
\end{table}

        As shown in Table \ref{tab: heatsink charactersitics}, only heatsink 4 passed the criteria for safe operation and therefore was the chosen heatsink. The threshold used provides a safe buffer in case higher power losses are seen. Therefore, heatsink 4 should be able to conduct heat losses at a sufficient rate even if the switching frequency is increased, keeping the chosen transistor within its operating region.

        