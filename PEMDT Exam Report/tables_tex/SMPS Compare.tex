\begin{table}[tbh!]
\caption{Fundamental differences between MOSFETs and IGBTs}
\label{tab: SMPS Comparison}
\resizebox{\columnwidth}{!}{%
\begin{tabular}{|p{2.1cm}|p{8cm}|p{8cm}|}
\hline
\textbf{} & \textbf{MOSFET} & \textbf{IGBT} \\ \hline
\textbf{Name} & Metal Oxide Semiconductor Field Effect Transistor & Insulated Gate Bipolar Transistor \\ \hline
\textbf{Structure} & MOSFETs consist of a gate electrode separated from the semiconductor channel by an insulating layer. When a voltage is applied to the gate, the channel controls the current flow between the source and drain terminals. & IGBTs consist of four layers: p+ collector, n- drift, n- buffer, and p+ emitter. The IGBT operates based on the voltage applied to the gate, controlling the current flow between the collector and emitter. The P-N-P-N structure allows the IGBT to combine the advantages of both MOSFETs and bipolar transistors. \\ \hline
\textbf{Conductivity} & MOSFETs are unipolar devices, meaning they primarily rely on the movement of either electrons (in n-channel MOSFETs) or holes (in p-channel MOSFETs) in the channel. The channel conductivity is modulated by the electric field generated by the voltage applied to the gate terminal. & IGBTs are bipolar devices, meaning they operate based on the movement of both electrons (in the n-type regions) and holes (in the p-type regions). The P-N-P-N structure allows for the flow of current through both electrons and holes. \\ \hline
\textbf{Voltage Rating} & MOSFETs are commonly used in low to medium-voltage applications. The voltage ratings of MOSFETs can range from a few volts to a few hundred volts. This is typical because they have lower voltage distribution and thinner depletion regions, given they have fewer layers. & IGBTs are designed for high-voltage applications. The voltage ratings of IGBTs range from a few hundred volts to several kilo-volts. IGBTs are suitable for applications that require high voltage blocking capability. \\ \hline
\textbf{Current Rating} & MOSFETs are generally used in low to medium current applications. The current ratings vary depending on the specific MOSFET model and design. & IGBTs have higher current handling capabilities compared to MOSFETs. They are capable of handling medium to high current levels. IGBTs are commonly used in high-power applications where significant current flow is required. \\ \hline
\textbf{Switching Speed} & MOSFETs offer faster switching speeds than IGBTs. Due to their unipolar nature, MOSFETs can quickly transition between on and off states, making them suitable for high-frequency applications. & IGBTs have slower switching speeds compared to MOSFETs. The presence of bipolar junction transistors causes a ``tail-off" effect when they turn off as the electrons and holes have to recombine; this results in longer turn-off times. \\ \hline
\textbf{Application Preferences} & MOSFETs are commonly used in low to medium-power applications. They find widespread use in consumer electronics, such as smartphones, laptops, and power supplies. They are also used in motor control, LED lighting, and other applications that require efficient power switching. & IGBTs are preferred in high-power applications. They are used in industrial drives, motor control systems, renewable energy systems, and high-power inverters. IGBTs excel in applications that require high voltage blocking, high current handling, and efficient power switching. They are well-suited for applications where power efficiency and reliability are key. \\ \hline
\end{tabular}%
}
\end{table}